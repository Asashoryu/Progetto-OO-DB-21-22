\chapter{Requisiti identificati}
Si progetterà una base di dati relazionale per la gestione di una rubrica telefonica avanzata. Il sistema memorizza e gestisce le informazioni relative ai contatti di un utente. Viene ammessa la possibilità per più persone di usare separatamente la propria rubrica, i cui contatti mantengono la completa indipendenza informativa rispetto a eventuali omonimi presenti in altre rubriche. Un indirizzo elettronico associa un contatto al suo account presso i sistemi di messaging. La Base di dati ignora il modo in cui le suddette informazioni esterne vengano reperite: si ammette che vengano recuperate dal dispositivo in uso, inserite nella Base di dati e, alla dichiarazione della corretta email, associate al contatto in questione. 
\section{Requisiti sui dati}
La Base di dati gestisce le seguenti classi di dati: 
\begin{itemize}
\item 
\textbf{Rubrica} indica gli utenti che hanno aggiunto una personale \rubrica. Il sistema consente di gestire le rubriche di più utenti;
\item 
\textbf{Elemento} indica l'atomo concettuale di ogni \Rubrica: senza almeno un \elemento è vuota. L'\elemento è una generalizzazione delle entità presenti in una \Rubrica, vale a dire i \contatti e i \gruppi;
\item 
\textbf{Contatto} indica gli \elementi di una \Rubrica che possiedono qualche numero di telefono. I numeri associati a un contatto possono essere più di uno e a ciascuno sono associabili possibili descrizioni (come: mobile, fisso, ufficio, ecc.). Proprio come per i numeri, si rappresentano gli indirizzi fisici e le email con record di attributi semplici. Altri attributi di un \contatto sono il nome, il secondo nome e il cognome;
\item 
\textbf{Gruppo} indica gli \elementi che sono collezione (anche vuote) di \contatti;
\item 
\textbf{Account} indica la collezione di informazioni associate a una email registrata su qualche sistema di messaging. Tali informazioni sono il nome del fornitore, l'indirizzo email, il nickname e la frase di benvenuto.
\end{itemize}

\section{requisiti sulle operazioni}
La Base di dati consente le seguenti operazioni sui dati:
\begin{itemize}
\item
Aggiunta e cancellazione di contatti e gruppi;
\item
Manipolazione delle informazioni di un \emph{Contatto} o di un \emph{Gruppo}; 
\item
Ricerca di \emph{contatti} per nome, email, account di messaging e per numero di telefono.
\end{itemize}