\chapter{Progettazione concettuale}
\section{Class Diagram}
{\Huge Foto}
        
\section{Analisi della ristrutturazione del Class Diagram}

       
\subsection{Analisi delle ridondanze}
Il Class Diagram non presenta ridondanze da eliminare. Tuttavia, al termine della ristrutturazione è possibile che alcune istanze di dati possano risultare ridondanti. Il sistema ammette che \contatti diversi possano condividere, concettualmente, numeri telefonici, indirizzi o, se in rubriche diverse, email uguali. La scelta progettuale di mantenere possibili duplicati per ogni \contatto è però inevitabile. Il funzionamento del modello relazionale impone che informazioni come il numero di telefono siano riscritte per ogni \contatto che ha lo stesso numero nella Base di dati. Si tratta di chiavi esterne che necessariamente metteranno in relazione un contatto con i suoi numeri di telefono. 
\subsection{Analisi degli identificativi}
Potenziali identificativi potrebbero essere gli attributi già presenti nelle classi. La scelta di un \emph{nome} come identificativo, tuttavia, è problematica dal punto di vista implementativo della Base di dati. Usare come identificativo di un contatto il suo nome, anche col secondo nome e cognome, vincolerebbe ogni utente che fa uso di una propria rubrica a evitare nomi che sono stati già scelti anche da contatti che fanno uso della propria rubrica. Problemi logici di questo genere sono risolvibili aggiungendo chiavi tecniche, o \emph{codici}, che fungano da identificativi. Si ragiona diversamente per la classe \Rubrica: è bene che non ci siano ambiguità tra gli utenti che hanno una propria \rubrica, pertanto sarebbe superfluo aggiungere codici. 
\subsection{Rimozione degli attributi multipli}
Nella classe \Contatto sono presenti i seguenti attributi multipli: Numeri, Email e Indirizzi. Tali attributi portano informazioni non eliminabili dalla Base di dati, ma possono essere anche aggiunti un numero arbitrario di volte. Queste ragioni sono sufficienti per suggerire la creazione di classi per ciascuno degli attributi multipli. Trattandosi anche di attributi multipli, la discussione è ripresa nella sezione successiva. 
\subsection{Rimozione degli attributi composti}
Gli attributi composti coincidono con gli attributi multipli. È naturale l'introduzione di classi per ciascuno degli attributi, con ciascuna classe avente come attributi i campi degli attributi composti. 
\subsection{Rimozione delle gerarchie}
La classe \Elemento rappresenta una generalizzazione delle classi \Contatto e \Gruppo. Si decide di rimuoverla poiché ha un mero valore concettuale.
\subsection{Partizione/Accorpamento delle associazioni}
Le associazioni individuate sono essenziali nell'indicare relazioni logiche tra le Entità. Si sceglie di non introdurre partizionamenti oppure accorpamenti.
\section{Class Diagram ristrutturato}
{\Huge Foto}
        
\section{Dizionario delle classi}
\begin{longtable}{p{0.1\textwidth}p{0.3\textwidth}p{0.5\textwidth}}
\toprule
Classe & Descrizione & Attributi                                                                                                                          \\ \midrule
\textbf{Rubrica} &
Descrittore delle rubriche aggiunte &
\verb|Utente_ID (varchar)|: Chiave tecnica. Identifica l’utente proprietario di una rubrica. 
\\ \midrule
\textbf{Gruppo} &
Descrittore di una collezione di contatti in rubrica &
\verb|Gruppo_ID (integer)|: Chiave tecnica. Identifica univocamente ciascun gruppo;\newline
Nome(varchar): Nome del gruppo.
\\ \midrule
\textbf{Contatto} &
Descrittore di un'entità contatto &
\verb|Contatto_ID (integer)|: Chiave tecnica. Identifica univocamente ogni contatto;\newline
Nome (varchar): indica il primo nome del contatto;\newline
SecondoNome (varchar):indica il secondo nome del contatto;\newline
Cognome (varchar):indica il cognome del contatto;\newline
Foto (blobs): immagine associata al contatto.
\\ \midrule
\textbf{Email} &
Descrittore di un entità indirizzo di posta elettronica di un contatto &
\verb|Email_ID (integer)|: Chiave tecnica. Identifica univocamente un'email di un contatto;\newline
IndirizzoEmail (varchar): stringa dell'email;\newline
Descrizione (varchar): informazione aggiuntiva di un email (privata, ufficio, ecc).
\\ \midrule
\textbf{Indirizzo} &
Descrittore di un entità indirizzo fisico di un contatto &
\verb|Indirizzo_ID (integer)|: Chiave tecnica. Identifica univocamente un indirizzo fisico di un contatto; \newline
Via (varchar): indica la via dell’indirizzo;\newline
Città (varchar): indica la città dell’indirizzo;\newline
Nazione(string): indica la nazione dell’indirizzo;\newline
CAP (string): indica il codice postale dell'indirizzo;\newline
Tipo(TipoA): indica se l’indirizzo è principale o secondario
\\ \midrule
\textbf{Telefono} &
Descrittore di un'entità numero telefonico di un contatto &
\verb|Telefono_ID (integer)|: Chiave tecnica. Identifica univocamente un numero di telefono di un contatto; \newline
Numero (varchar): stringa del numero;\newline
Descrizione (varchar): informazione aggiuntiva di un numero di telefono (mobile, fisso, ufficio ecc).
\\ \midrule
\textbf{Account} &
Descrittore delle informazioni recuperate da un account &
\verb|Account_ID (varchar)|: Chiave tecnica. Identifica l’utente proprietario di una rubrica;\newline
Fornitore (varchar): indica il fornitore da cui è recuperato l'account (Whatsapp, Telegram, Teams, ecc..);\newline
IndirizzoEmail (varchar): indirizzo email associato a un account;\newline
FraseStato (varchar): frase di benvenuto associata a un account;\newline
Nickname(varchar): indica il nickname associato a un account.
\\ \bottomrule
\end{longtable}
\section{Dizionario delle associazioni}
\begin{longtable}{p{0.3\textwidth}p{0.3\textwidth}p{0.3\textwidth}}
\toprule
Associazione & Descrizione & Classi coinvolte
\\ \midrule
\textbf{AppartenenzaGruppo} &
Esprime l'appartenenza di un gruppo a una rubrica &
Rubrica [1]: indica la rubrica che contiene un gruppo;\newline
Gruppo[0..*]: indica il gruppo appartenente a una rubrica.
\\ \midrule
\textbf{AppartenenzaContatto} &
Esprime l'appartenenza di un contatto a una rubrica &
Rubrica [1]: indica la rubrica che contiene un contatto;\newline
Contatto[0..*]: indica il contatto appartenente a una rubrica.
\\ \midrule
\textbf{Composizione} &
Esprime l'appartenenza di un contatto a un gruppo &
Gruppo [0...*]: indica il gruppo che contiene un contatto;\newline
Contatto[0..*]: indica il contatto appartenente a un gruppo.
\\ \midrule
\textbf{Associa} &
Esprime l'associazione di un account a un contatto &
Contatto [0...*]: indica il contatto a cui è associato un account;\newline
Account [0..*]: indica l'account che è associato a un contatto.
\\ \midrule
\textbf{RecapitoEmail} &
Esprime l’appartenenza di una email ad un contatto &
Email [0..*]: indica un'entità email appartenente ad un contatto;\newline
Contatto[1]: indica il contatto a cui è associata l'entità email.
\\ \midrule
\textbf{RecapitoIndirizzo} &
Esprime l’appartenenza di un'entità indirizzo fisico ad un contatto &
Indirizzo [1..*]: indica un'entità indirizzo fisico appartenente ad un contatto;\newline
Contatto[1]: indica il contatto a cui è associata l'entità indirizzo.
\\ \midrule
\textbf{RecapitoTelefono} &
Esprime l’appartenenza di un'entità numero di telefono ad un contatto &
Telefono [1..*]: indica un'entità numero di telefono appartenente ad un contatto;\newline
Contatto[1]: indica il contatto a cui è associata l'entità numero di telefono.
\\ \midrule
\textbf{AppartenenzaContatto} &
Esprime l'appartenenza di un contatto a una rubrica &
Rubrica [1]: indica la rubrica che contiene un contatto;\newline
Contatto[0..*]: indica il contatto appartenente a una rubrica.
\\ \bottomrule
\end{longtable}